\ehi belongs to a broad class of vulnerabilities known as command injection vulnerabilities. However, unlike its more popular siblings, SQL injection~\cite{sql1, sql0, sql2}, Cross-Site Scripting~\cite{Injection1, KleinAmit}, or HTTP Header Injection~\cite{sessionride}, relatively little research is available on \ehi vulnerabilities.

As with other vulnerabilities in this class, \ehi is caused due to improper or nonexistent sanitization of user input. If the program that constructs \emails from user input fails to check for the presence of \email headers in the user input, a malicious user---using a well-crafted payload---can control the headers set for this particular e-mail. \ehi vulnerabilities can be leveraged to enable malicious attacks, including, but not limited to, spoofing or phishing.
