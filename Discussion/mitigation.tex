\subsection{Mitigation Strategy}
\label{disc:mitigation}
After demonstrating that \ehi vulnerabilities exist on the web at large, we now describe the most common measures that can be taken to prevent the occurrence of this vulnerability, or at least reduce the impact.
\begin{itemize}
	\item Use Mail Libraries\\
	Using a safe \email library is the preferred way of preventing \ehi vulnerabilities. Using a library that is well tested can remove the burden of input sanitization from the developer. 
	A list of known secure libraries for each language and framework discussed previously is shown in Table~\ref{tab:maillib}.
	
	Using libraries such as PEAR Mail, PHPMailer, Apache Commons E-Mail, Contact Form 7, and Swiftmailer can significantly reduce the occurrence of \ehi vulnerability.
	\begin{table}[tbp]
		\centering
		\begin{tabular}{|l|l|}
			\hline
			\multicolumn{1}{|c|}{\textbf{Language}} &
			\multicolumn{1}{c|}{\textbf{Mail Libraries}} \\
			\hline
			PHP & {{PEAR Mail\tablefootnote{PEAR Mail Website: https://pear.php.net/package/Mail}, PHPMailer\tablefootnote{PHPMailer Website:\\ https://github.com/PHPMailer/PHPMailer}, Swiftmailer\tablefootnote{Swiftmailer Website: http://swiftmailer.org/}}}\\
			\hline
			Python & SMTPLib with email.header.Header\tablefootnote{instead of using email.parser.Parser to parse the header}\\
			\hline
			Java & Apache Commons E-Mail\tablefootnote{Apache Commons E-Mail: https://commons.apache.org/proper/commons-email/}\\
			\hline
			Ruby & Ruby Mail \textgreater{}= 2.6\tablefootnote{Ruby Mail Website: https://rubygems.org/gems/mail}\\
			\hline
			WordPress & Contact Form 7\tablefootnote{Contact Form 7 Download: https://wordpress.org/plugins/contact-form-7/}\\
			\hline
		\end{tabular}
		\caption[\titlecap{Mail libraries that prevent e-mail header injection}]{Mail libraries that prevent e-mail header injection.}
		\label{tab:maillib}
	\end{table}
	\item Use a Content Management System (CMS) \\
	Content management systems such as WordPress and Drupal include libraries and plugins to prevent \ehi. Thus, websites built with such CMS' are usually resistant to these attacks. However, it is advised to use the correct \email plugin, as not all plugins might be secure.
	An example of a secure plug-in is included as part of Table~\ref{tab:maillib}.
	
	\item Input Validation\\
	If neither of the two options are feasible, due to reasons such as the website being an in-house production, or due to lack of support infrastructure, developers can choose to perform proper input sanitization. Sanitization should be done keeping in mind RFC5322~\cite{rfc5322}, and care must be taken to ensure that all edge cases are taken into account.
	
	%% Client Side validation alone is not sufficient, and must be supplemented by server-side validation to mitigate the attack. Constant updates to validation methods are required so that new attack vectors do not harm the website in any way.
	%% Test driven development for such validation methods is also encouraged so that we can be reasonably sure of our defense mechanisms.
\end{itemize}
