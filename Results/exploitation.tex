\subsection{Exploitation Evidence}
We compared the \ips IPs that our system found to be vulnerable to \ehi against 13 well-known IP blacklists, to see if these IPs were being exploited by attackers to send spam. The blacklists that we used were: 
\texttt{zen.spamhaus.org},
\texttt{spam.abuse.ch},
\texttt{cbl.abuseat.org},
\texttt{virbl.dnsbl.bit.nl},
\texttt{dnsbl.inps.de},
\texttt{ix.dnsbl.manitu.net},
\texttt{dnsbl.sorbs.net},
\texttt{bl.spamcannibal.org},
\texttt{bl.spamcop.net},
\texttt{dnsbl-1.uceprotect.net},
\texttt{dnsbl-2.uceprotect.net},
\texttt{dnsbl-3.uceprotect.net},
\texttt{db.wpbl.info}

We found that \ipsblacklist of these IPs were blacklisted on at least one of the above blacklists for sending out spam, and \ipsblacklistmulti of them were found on multiple blacklists. We do not have enough data to make an observation about whether these attackers are exploiting \ehi to send out the spam, as an alternative hypothesis is that these IPs are on the blacklists because the server has different vulnerabilities that attackers exploit to cause the server to send spam (assuming that the server is normally benign).  
